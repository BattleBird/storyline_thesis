\documentclass[a4paper]{article}
\usepackage[T1]{fontenc}
\usepackage{txfonts}

\begin{document}
		\textbf{STORYLINE}\\
	\textit{A. Introduction}
	\begin{itemize}
		\item Overview
		\begin{itemize}
			\item The Ethereum Virtual Machine is responsible for executing the bytecode.
			\item The implementations of EVM in different programming languages may have \textbf{defects}.
			\item Many efforts are put into the validation and vulnerabilities detection on smart contract. Few techniques exist to help systematically validate EVMs. 
			\item Existing work, EVMfuzzer, mutates the smart contracts and then use the Solidity compiler to compile the mutants into bytecode fed into differently testing EVMs. However, bytecode generated by Solidity compiler is syntactically valid. Different EVM platforms may deal with the validity of bytecode in a different way. We observe that bytecode-level mutating can directly generate some syntactically invalid bytecodes such as the length of bytecode is odd. Normally, the length of bytecode generated by Solidity compiler is even.
			\item Inspired by one notable effort, classming, alters the \textbf{control flow} of Java bytecode for JVMs testing. EVMs may execute uncommon
			bytecode instruction sequences in different manners.
			
			\item To perform
			efficient EVM testing, the difficulty lies in synthesizing inputs that exercise a \textbf{variety} of code paths in the EVM. That is to say, how can we quickly select the seeds from a large number of mutants to perform the next iterations of mutans generation while ensuring that the resulting mutants can explore the EVM's execution paths as much as possible.

			\item This paper tackles this problem by introducing a \textit{\textbf{coverage-directed fuzzing}} approach that focuses on representative bytecodes for differential testing of EVMs. Our core idea is to continuously (1)manipulate the \textbf{control flow} of seed bytecodes of smart contracts to generate semantically different mutants and (2)execute the mutants on a \textit{reference} EVM and use \textit{\textbf{coverage uniqueness}} as a guideline to selectively accept the generated mutants to steer the mutation process toward diverse mutants. The generated mutants are then employed to differentially test EVMs.
		\end{itemize}
		\item Contributions
		\begin{itemize}
			\item EVM testing requires abundant runable, diverse \textit{semantically valid} test \textbf{bytecodes}. Our work is \textbf{the first} to develop a tool that aims at systematically altering the \textbf{control flow} of bytecode to generate such tests.
			\item We integrate the \textbf{Coverage-guided fuzzing} algorithm to the mutation process for effective differential EVM testing. Given a seed, our approach can create a large number of test bytecodes with \textbf{diverse semantics} to help expose EVM implementation differences.
			\item We have evaluated our framework on mature and popular Ethereum Virtual Machines such as \textbf{go-ethereum}, \textbf{parity-ethereum} and \textbf{aleth}. We have analyzed the \textbf{crashes} and \textbf{differences}. Accompanied by manual root cause analysis, we found some previously unknown issues.
		\end{itemize}
		\item Thesis Arrangement
		\begin{itemize}
			\item Chapter2 introduces background knowledge  including Ethereum Virtual Machine, smart contract, opcodes, bytecode and an motivating example. 
			\item Chapter3 introduces our approach.
			\item Chapter4 introduces the implementation and evaluation.
			\item Chapter5 introduces the discussions.
			\item Chapter6 introduces related works about coveraged-guided fuzzing, differential testing, Java Virtual Machines testing and Ethereum Virtual Machines Testing.
			\item Chapter7 shows the conclusion.
		\end{itemize}
	\end{itemize}
	
	\textit{B. Preliminaries and Motivating Examples}
	\begin{itemize}
		\item Ethereum Virtual Machine
		\begin{itemize}
			\item  There are tens
			of thousands of transactions happened via the clients based on
			different EVM implementations everyday. Therefore, the vulnerability
			hidden in any EVM version might result in serious consequences such as the financial loss.
		\end{itemize}
		\item Smart Contract, Opcodes and Bytecode
		\begin{itemize}
			\item Smart contracts are compiled by Solidity compiler to low-level machine instructions(called \textbf{opcodes}). The EVM uses a set of opcodes to execute specific tasks. Especially, this paper focus on \textbf{Program counter related opcodes} (JUMP, JUMPI, JUMPDEST). And we use these instructions to design mutators. It has several reasons. Firstly, these opcodes can alter a program's control--flow. Secondly, these mutators can generate a large number of mutants in a short time. Thirdly, 
			the more public functions in a smart contract, and the more complex the structure of the function, the more mutants that can be generated. Therefore, our approach do not need too many seed contracts. We only need to select enough contracts with more inner public function and rich structure in these functions.
			\item In order to efficiently store opcodes, \textbf{opcodes} are encoded to \textbf{bytecode}
			\item Bytecode is part of the input fed into the EVM.
		\end{itemize}
		\item{An Illustrative Example}
		\begin{itemize}
			\item \textbf{infinite loop} can cause two set of different behaviours
			\begin{itemize}
				\item mutator3 + mutator4
				\item mutator2
			\end{itemize}
		\end{itemize}
	\end{itemize}
	\textit{C. Design of Approach}
	\begin{itemize}
		\item Overview
		\begin{itemize}
			\item Given a set of seed bytecode files \textit{f}, our approach aims at creating many mutants of files in \textit{f} by manipulating \textit{f's} bytecode, 
			\item and then these mutants are executed in test program. After execution, we collect the \textit{branch coverage} of each mutants, and use the \textit{unique branch coverage} as a guidline to choose the resulting mutants for the next iteration.
			\item The generated bytecode mutants are then employed to differentially test EVMs
			\item The whole process is splited into following parts, including opcode and function block generation, abnormal opcodes generation, translate opcodes into bytecode, CGF guide the seed selection and unified EVM execution.
		\end{itemize}
		\item Opcode and Function Block Generation
		\begin{itemize}
			\item We use the \textit{disassembler} in \textbf{Vandal} to convert seed bytecode to opcodes and \textit{decompiler} to generate the the byteocde's function blocks for further mutation
		\end{itemize}
		\item Abnormal Opcodes Generation
		\begin{itemize}
			\item extract body blocks of each public function
			\item extract all addresses of JUMP or JUMPI insturctions
			\item mutate the bytecode by a set of \textit{pre-defined mutators}
		\end{itemize}
		\item Translate Opcodes into Bytecode
		\begin{itemize}
			\item After we mutate the opcodes, we need to translate the opcodes into bytecode, which is one of the input arguments for Ethereum Virtual Machine.
		\end{itemize}
		\item CGF Guide the Seed Selection
		\begin{itemize}
			\item In each iteration, new inputs are generated by running the mutation program and the test program is then executed on each input. 
			\item During the execution, the \textbf{branch coverage} of each seed will be recorded. If the corresponding input covers a \textbf{new coverage point} that has not been exercised by any previous valid input, then the input is added to the seed pool for the next iteration
		\end{itemize}       	
		\item Unified EVM Execution
		\begin{itemize}
			\item The first step is to get the data that can directly feed into EVM
			platform, namely the contract runtime bytecode and input parameters.
			\item The second step is to call the execution interface of each EVM to execute the
			contract data, standardize the output format of the execution result
			and save the output.
			\item The third step is to analyze the output files of each platform and the crashes during the execution to uncover potential bugs.
		\end{itemize}
	\end{itemize}
	\textit{D. Implementation and Evaluation} 
	\begin{itemize}
		\item overview
		\begin{itemize}
			\item In this chapter, we present the implementation details of our algorithm. Our approach has conducted large-scale mutations on 52 real-world smart contracts and several EVM discrepancies and vulnerabilities have been found. We answer the following three questions: (i) Is there any
			inconsistency among EVMs? (ii) Could our approach achieve higher
			branch coverage? (iii) Is it possible to
			find EVM bugs through our approach?
		\end{itemize}
		\item Data and Environment Setup
		\begin{itemize}
			\item with 8 cores (Intel
			i7-4870HQ @2.50GHz), 16GB of memory, and Ubuntu 16.04.6 LTS as the
			host operating system. The 52 real-world contracts we used
			for mutation were crawled from the Etherscan and we obtained the corresponding runtime bytecode and abi function via the Solidity
			language compiler solc 0.6.2. EVMFuzz tested three widely
			used EVM platforms for each mutated contract, that is, go-evm,parity-evm, aleth-vm
		\end{itemize}
		\item Is there any
		inconsistency among EVMs?
		\begin{itemize}
			\item \textbf{gasUsed comparison}.Explaining why the different platforms have different gas consumption
			\item \textbf{error information inconsistency}. Explaining why the different platforms have different error reports.
			\item \textbf{Memory occupation and executed result inconsistency}. Explaining why the different platforms have different executed results.
		\end{itemize} 
		\item Could our approach achieve higher
		branch coverage?
		\begin{itemize}
			\item the branch coverage in our approach grow faster than EVMFuzzer in the first three iterations which shows that the \textbf{coverage-guided fuzzing} can explore the code paths quickly.
		\end{itemize}
		\item Is it possible to
		find EVM bugs through our approach?
	\end{itemize}
	\textit{E. Discussion}
	\begin{itemize}
		\item \textbf{Mutators design}. 
		\begin{itemize}
			\item Considering the characterics of control-flow, we designed 5 mutators,
			whose quantity and qualities still need improving. Currently, we just created some mutators based on JUMP, JUMPI, JUMPDEST opcodes and their targets address. In the future, we
			will make more semantic level mutations based on bytecode, taking the optimization of bytecode into consideration.
		\end{itemize}
		\item \textbf{More efficient fitness function}
		\begin{itemize}
			\item In this paper, we use
			\textbf{unique branch coverage} as the fitness function. For instance, if a test case
			covers a new branch in the program under test, then we add it to
			the pool as it meets the fitness criterion.Despite its wide use, it suffers from critical information loss~. In the future, we will improve the fitness function, considering the relationships between program executions.
		\end{itemize}
		
		\item \textbf{Testing more EVM  implementation}. 
		\begin{itemize}
			\item The core idea of
			differential testing is testing on at least two objects with the
			same test inputs. Therefore, we analyze the differences in three
			EVM platforms: geth, parity-EVM, aleth. However, in
			the real world, many platforms based on other programming
			languages are also widely used, and they may also have some
			potential vulnerabilities in implementation. We will further incorporate them into our framework.
		\end{itemize}
	\end{itemize}
	\textit{F. Related Work}
	\begin{itemize}
		\item Coverage-Guided Fuzzing
		\begin{itemize}
			\item CGF instruments the test program to provide dynamic feedback in the form of \textit{code coverage} for each run. If the mutated inputs can lead to new \textbf{branch coverage} in the test program, they are saved for next fuzzing iteration.
		\end{itemize} 
		\item Differential Testing
		\begin{itemize} 
			\item Differenetial testing attempts to detect bugs by providing the same input to a series of similar test programs
		\end{itemize}
		\item Java Virtual Machine testing
		\begin{itemize}
			\item Chen et al. 
			achieves JVM implementations testing via systematic manipulation of the control-
			and data-flow of live bytecode to effectively testing verifiers and execution engines at the backend in the JVM.
		\end{itemize}
		\item Ethereum Virtual Machine testing
		\begin{itemize}
			\item EVMFuzzer directly mutated smart contracts to differentially testing EVMs. However, seed contracts generated by EVMFuzzer has no syntax error and are not diverse.
		\end{itemize}
		\item Our main difference
		\begin{itemize}
			\item Different from the above work, our work
			mainly focuses on discovering the defects and vulnerabilities in
			EVM by mutating the \textbf{bytecode} of smart contracts. In the case that most researchers are concerned about smart
			contracts validation and no researchers focus on the bytecodes when testing the EVM, motivated by this, we proposed our approach which aims at manipulating the \textbf{control-flow} of bytecodes and integrated the \textbf{CFG} into \textbf{differential testing}, using the \textit{branch coverage uniqueness} as the \textit{fitness function} to select the diverse seeds.
		\end{itemize}
	\end{itemize}
	\textit{G. Conclusion}
	\begin{itemize}
		\item In this paper, we presented a technique that incorporates ideas from coverage-guided fuzzing into differential testing, the first bytecode level automated fuzzing framework, to effenciently find vulnerabilities of EVM implementations. 
		\item We design 5 mutators related to changing the control-flow of bytecode and use the \textit{branch coverage uniqueness} as the guideline during the fuzzing to explore the scode paths to generate more diverse input tests.
		\item We evaluated our approach on 3 different 
		EVM implementations, we found that there exists gasUsed inconsistencies and error information inconsistencies among EVMs, our approach achieved higher
		branch coverage than baseline techniques and our generated corner cases could find some previously unknown problems. 
		\item Our future work mainly
		includes developing more specific muators which can generate more corner cases to uncover the vulnerabilities, and conducting more
		extensive evaluations on more EVMs.
	\end{itemize}
\end{document}
